\section{Conclusions}
\label{sec:conclusions}

The main contribution of this paper is the formalization of a natural
notion, personalized flexible context extraction, that corresponds to
a need that all news consumers have felt to some extent or the
other. We have outlined the structure of a system that is geared to
provide a richly featured news browsing experience that revolves
around the notion of context. Using the news graphs first described by
Choudhary et. al.~\cite{choudhary@ecir2008} we have shown how such a
news browsing experience can be achieved. Our news browsing tool,
ESTHETE, attempts to provide the kind of context-aware news
consumption environment that makes it possible for a lay user, or an
expert, to explore the various aspects of a story to whatever level of
detail required, and does not tie down the user to particular
structures extracted from the corpus. The fundamental difference
between this work and the literature preceding it is that earlier
papers took the approach of developing algorithms for processing the
corpus to respond to certain classes of queries whereas we approach
news browsing as a data structuring problem, preprocessing the corpus
into a structured entity that contains the complex relations that a
human being needs to unravel the context of a particular news
story. Our user studies have shown that ESTHETE addresses the need for
context without sacrficing on the need for current information and
overall provides complex features in an easy-to-use interface.

